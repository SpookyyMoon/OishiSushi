\documentclass[a4paper,12pt]{article}

\usepackage[spanish]{babel}
\usepackage[utf8]{inputenc}
\usepackage[T1]{fontenc}
\usepackage{lmodern}
\usepackage{graphicx}
\usepackage{geometry}
\usepackage{hyperref}
\usepackage{tabularx}
\usepackage{listings}
\usepackage{xcolor}

\graphicspath{ {./imagenes/} }

\geometry{margin=2.5cm}
\begin{document}
    \begin{titlepage}
    \centering
    {\bfseries\LARGE Oishii Sushi (APP Móvil y Escritorio)\par}
    \vspace{2cm}
    {\scshape\Huge Documentación de Arquitectura de Software\par}
    \vspace{5cm}
    \vfill
    {\Large Jacobo Pérez de Torres\par}
    \vfill
    \end{titlepage}

    \tableofcontents
    \newpage

    \section{Introducción}
    Oishii Sushi es un conjunto de aplicaciones (Móvil y Escritorio) desarrollada como práctica para las asignaturas de Acceso de Datos, Programación multimedia y móvil y Desarrollo de Interfaces.
    La aplicación móvil permite seleccionar una mesa del restuarante, hacer pedidos desde la carta y tramitar el pago de la cuenta. La aplicación de escritorio, recibe las comandas de los pedidos realizados
    desde la APP y tramitarlas.
    Ambas aplicaciones se comunican a traves de un backend con Express y MongoDB Atlas.
    \section{Arquitectura de la Aplicación}
    La aplicación se organiza siguiendo una arquitectura por capas que separa la interfaz de usuario, la lógica de negocio y el manejo de datos (Similar al modelo vista-controlador).
    \begin{itemize}
        \item \textbf{Capa de interfaz (UI):} Activities de Android Studio en XML y vistas de JavaFX FXML.
        \item \textbf{Capa de lógica:} Gestiona el envío de datos entre vistas y activities, la lógica del negocio (Platos, comandas, carrito, cuenta...) y coordina la base de datos.
        \item \textbf{Capa de datos:} Contiene el acceso a la base de datos y a la API.
    \end{itemize}

    \section{Diseño de la base de datos (MongoDB)}
    En el backend se definen las siguientes colecciones de Items:
    
    \vspace{0.5cm}

      - Comandas:
        \begin{lstlisting}
        numeroMesa: Number,
        pedidoPlatos: Array,
        atendidaComanda: Boolean
        \end{lstlisting}
        
      - Platos:
        \begin{lstlisting}
        nombrePlato: String,
        precioPlato: Number,
        categoriaPlato: String,
        unidadesPlato: Number    
        \end{lstlisting}
            
      - Mesas:
        \begin{lstlisting}
        numeroMesa: Number,
        ocupadaMesa: Boolean,
        carritoMesa: Array
        \end{lstlisting}
    \newpage
    \section{API}
    La API sigue un modelo RESTful. Con los siguientes endpoints:

    \vspace{0.5cm}

    \begin{center}
        \begin{tabularx}{0.9\textwidth}{|X|l|X|X|}
            \hline
            \textbf{Método} & \textbf{Endpoint} & \textbf{Descripción} & \textbf{Parámetros} \\
            \hline
            GET & /platos & Recibe la lista de platos & N/A\\
            \hline
            GET & /comandas & Recibe la lista de comandas & N/A\\
            \hline
            GET & /mesas & Recibe la lista de mesas & N/A\\
            \hline
            POST & /platos & Añade un nuevo plato (Sin uso actual) & N/A\\
            \hline
            POST & /comandas & Añade una nueva comanda & N/A\\
            \hline
            POST & /mesas & Añade una nueva mesa & N/A\\
            \hline
            PUT & /platos & Actualiza un plato (Sin uso actual) & N/A\\
            \hline
            PUT & /comandas/:numeroMesa & Actualiza la información de una comanda & numeroMesa\\
            \hline
            PUT & /mesas/:numeroMesa & Acutaliza la información de una mesa & numeroMesa\\
            \hline
        \end{tabularx}
    \end{center}

    \newpage
    \section{Diagrama de flujo de la APP}
    A traves del siguiente diagrama se muestra el flujo de la aplicación móvil:

    \vspace{0.5cm}

    \begin{center}
        \includegraphics[scale=0.4]{imagen1.png}
    \end{center}

    \section{Tecnologías utilizadas}
    Para el desarrollo de la aplicación se han empleado las siguientes tecnologías:

    \subsection{Frontend App móvil (Android Studio)}
    \begin{itemize}
      \item Lenguaje: Java.
      \item Interfaz: XML.
      \item Conexión con backend: Retrofit.
    \end{itemize}
    \subsection{Frontend Desktop (JavaFX)}
    \begin{itemize}
      \item Lenguaje: Java.
      \item Interfaz: FXML.
      \item Conexión con backend: Retrofit.
    \end{itemize}
    \subsection{Servidor Backend}
    \begin{itemize}
      \item Lenguaje: JavaScript (Node.js).
      \item Framework: Express.js.
      \item Base de datos: MongoDB.
      \item ODM: Mongoose.
    \end{itemize}
    \section{Clases e Interfaces}
    En esta sección se muestra una lista de todas las clases existentes.

    \subsection{Clases principales (Móvil y Escritorio)}
    A continuación se describen las principales clases e interfaces que componen la aplicación:

    \begin{itemize}
        \item \textbf{Comandas} (entidad)
        \begin{itemize}
            \item Contiene los atributos de una comanda: int numeroMesa, \texttt{List<Platos>} pedidoPlatos, boolean atendidoComanda.
            \item Representa el modelo de datos de las comandas de la aplicación.
            \item En la aplicación de escritorio esta clase varía añadiendo en atributos el id de mongo, para poder actualizar cada comanda.
        \end{itemize}

        \item \textbf{Platos} (entidad)
        \begin{itemize}
            \item Contiene los atributos de una comanda: String nombrePlato, double precioPlato, String categoriaPlato, int unidadesPlato.
            \item Representa el modelo de datos de los platos de la aplicación.
        \end{itemize}

        \item \textbf{Mesas} (entidad)
        \begin{itemize}
            \item Contiene los atributos de una comanda: int numeroMesa, boolean ocupadaMesa, \texttt{List<Comandas>} comandasMesa, \texttt{List<Platos>} carritoMesa.
            \item Representa el modelo de datos de las mesas de la aplicación.
        \end{itemize}

        \item \textbf{Mesas} (entidad)
        \begin{itemize}
            \item Contiene los atributos de una comanda: int numeroMesa, boolean ocupadaMesa, \texttt{List<Comandas>} comandasMesa, \texttt{List<Platos>} carritoMesa.
            \item Representa el modelo de datos de las mesas de la aplicación.
        \end{itemize}


        \item \textbf{APIService} (servicio)
        \begin{itemize}
            \item Conecta con los endpoints del backend para establecer los metodos de tratado de datos.
        \end{itemize}


        \item \textbf{ApiAdapter} (adaptador)
        \begin{itemize}
            \item Se conecta con la URL del backend a traves de retrofit,
        \end{itemize}
    \end{itemize}

    \subsection{Clases exclusivas de Android Studio}
    \begin{itemize}
        \item {\textbf{MainActivity}}
        \begin{itemize}
            \item Controlador del layout \texttt{activity\_main.xml}
            \item Inicia la aplicación y permite acceder a selección mesas.
        \end{itemize}

        \item {\textbf{SeleccionMesa}}
        \begin{itemize}
            \item Controlador del layout \texttt{seleccion\_mesa.xml}
            \item Permite la selección de una mesa, comprueba cuáles están ocupadas, regula el acceso a mesas ocupadas y envía al usuario a Carrito.
        \end{itemize}

        \item {\textbf{Carta}}
        \begin{itemize}
            \item Controlador del layout \texttt{carta.xml}
            \item Permite la selección de platos, retroceder para liberar la mesa y desplazarse al carrito.
        \end{itemize}

        \item {\textbf{ContenedorPlatosAdaptador}}
        \begin{itemize}
            \item Adaptador que controla el RecycleView de la clase Carta.
        \end{itemize}

        \item {\textbf{Carrito}}
        \begin{itemize}
            \item Controlador del layout \texttt{carrito.xml}
            \item Permite visualizar el carrito con sus correspondientes platos y hacer un pedido.
        \end{itemize}

        \item {\textbf{ContenedorCarritoAdaptador}}
        \begin{itemize}
            \item Adaptador que controla el RecycleView de la clase Carrito.
        \end{itemize}

        \item {\textbf{CarritoVacio}}
        \begin{itemize}
            \item Controlador del layout \texttt{carrito\_vacio.xml}
            \item Permite visualizar el carrito vacio y regresar a la carta.
        \end{itemize}

        \item {\textbf{Cuenta}}
        \begin{itemize}
            \item Controlador del layout \texttt{ver\_cuenta.xml}
            \item Permite visualizar la cuenta total de pedidos, pagar y salir del restaurante.
        \end{itemize}

        \item {\textbf{ContenedorCuentaAdaptador}}
        \begin{itemize}
            \item Adaptador que controla el RecycleView de la clase Cuenta.
        \end{itemize}
    \end{itemize}

    \subsection{Clases exclusivas de JavaFX}
    \begin{itemize}
        \item {\textbf{HelloController}}
        \begin{itemize}
            \item Controlador del layout principal, contiene todos los métodos y funciones.
        \end{itemize}
    \end{itemize}

  \section{Problemas encontrados}
  Durante el desarrollo de las aplicaciones móvil y de escritorio se encontraron los siguientes problemas:

  \section{Problemas encontrados}
  Durante el desarrollo de las aplicaciones móvil y de escritorio surgieron varios problemas, algunos de ellos bastante curiosos y otros más técnicos, que fueron solucionados de diferentes maneras:

    \begin{itemize}
        \item \textbf{Actualización de datos en la API:} Al principio los cambios realizados en la aplicación no se reflejaban en el backend, lo que generaba confusión al probar los pedidos.  
        \textit{Solución:} Aprendí a usar correctamente los métodos PUT y PATCH con Retrofit para que los objetos se sincronizaran correctamente (\cite{futurestud, geeksforgeeks, programacionymas}).

        \item \textbf{Carga de imágenes en RecyclerView:} Las imágenes de los platos a veces no se cargaban o se veían tarde, haciendo que la interfaz pareciera lenta.  
        \textit{Solución:} Implementé carga asíncrona en los ViewHolders para que la UI no se bloquease y todo se mostrara correctamente (\cite{stackoverflow}).

        \item \textbf{Eventos onClick en RecyclerView:} Al principio, los clics en los elementos personalizados no funcionaban bien.  
        \textit{Solución:} Moví los onClickListener dentro del adaptador y los pasé al Activity correspondiente para que respondieran correctamente (\cite{stackoverflow, github}).

        \item \textbf{Compatibilidad de objetos en Android Studio:} Intentar pasar objetos entre Activities daba errores de tipo “no se puede castear a Parcelable”.  
        \textit{Solución:} Implementé correctamente la interfaz \texttt{Parcelable} en los objetos que necesitaba transferir (\cite{stackoverflow}).

        \item \textbf{Conexión de JavaFX con NodeJS:} La aplicación de escritorio no lograba conectarse bien con la API, lo que hacía que los datos no se actualizaran.  
        \textit{Solución:} Usé Retrofit también en JavaFX y gestioné correctamente los hilos para evitar bloqueos (\cite{chatgpt, stackoverflow}).

        \item \textbf{Visibilidad de elementos en Android Studio:} Algunos botones se veían bien en el editor gráfico pero no aparecían en el dispositivo al ejecutar la app.  
        \textit{Solución:} Comprobé las restricciones, la jerarquía de layouts y que los IDs coincidieran correctamente (\cite{stackoverflow, youtube}).

        \item \textbf{Sincronización con MongoDB:} Me costó actualizar arrays de objetos y mantener los datos de pedidos y mesas sincronizados entre backend y frontend.  
        \textit{Solución:} Aprendí a usar correctamente los métodos de MongoDB Realm y probé varias veces PUT y POST hasta que los cambios se guardaban correctamente (\cite{mongodb}).
    \end{itemize}

    \section{Bibliografía}
    Para llevar a cabo el desarrollo de esta documentación y siendo la primera vez que tenía que simular una API inexistente y documentar la arquitectura de un
    software, he necesitado consultar diversas fuentes, desde StackOverflow hasta consultas sobre Latex a chatgpt. Todas las fuentes empleadas se citan a continuación:

    \begin{thebibliography}{9}

    \bibitem{programacionymas}
    Programacionymas. \textit{Consumir una API usando retrofit}. 
    Disponible en: \url{https://programacionymas.com/blog/consumir-una-api-usando-retrofit}.

    \bibitem{geeksforgeeks}
    Geeksforgeeks. \textit{Como actualizar datos en una API usando retrofit en Android Studio}. 
    Disponible en: \url{https://www.geeksforgeeks.org/android/how-to-update-data-in-api-using-retrofit-in-android/}.

    \bibitem{stackoverflow}
    StackOverflow. \textit{Como fijar una imagen en un view holder de forma asíncrona}. 
    Disponible en: \url{https://stackoverflow.com/questions/33575011/how-to-set-image-to-a-view-holder-asynchronously}.

    \bibitem{stackoverflow}
    StackOverflow \textit{Como definir una forma circular en Android Studio a traves de un archivo xml drawable}. 
    Disponible en: \url{https://stackoverflow.com/questions/3185103/how-to-define-a-circle-shape-in-an-android-xml-drawable-file}.

    \bibitem{stackoverflow}
    StackOverflow. \textit{Como implementar un onClickListener en un viewholder customizado en mi RecyclerView}. 
    Disponible en: \url{https://stackoverflow.com/questions/46640897/how-to-implement-onclicklistener-for-custom-viewholder-in-my-recycler-view}.

    \bibitem{stackoverflow}
    StackOverflow. \textit{Eliminar y refrescar en un RecyclerView}. 
    Disponible en: \url{https://stackoverflow.com/questions/68780473/delete-and-refresh-in-recyclerview}.

    \bibitem{stackoverflow}
    StackOverflow. \textit{JavaFX Como hacer una imagen clickeable en Scene Builder}. 
    Disponible en: \url{https://stackoverflow.com/questions/40495658/javafx-how-make-a-clickable-image-using-scenebuilder}.

    \bibitem{stackoverflow}
    StackOverflow. \textit{Como conectar un cliente de JavaFX a un servidor de NodeJS}. 
    Disponible en: \url{https://stackoverflow.com/questions/58750418/connecte-javafx-client-to-a-nodejs-server}.

    \bibitem{stackoverflow}
    StackOverflow. \textit{Android Studio - Objeto no puede ser casteado a Parceable}. 
    Disponible en: \url{https://stackoverflow.com/questions/41924068/object-cannot-be-cast-as-parcelable}.

    \bibitem{stackoverflow}
    StackOverflow. \textit{Android Studio - Como cambiar el color del borde de un SearchView}. 
    Disponible en: \url{https://stackoverflow.com/questions/36888511/change-border-color-of-a-searchview}.

    \bibitem{stackoverflow}
    StackOverflow. \textit{Android Studio - Ejemplo simple de un RecyclerView}. 
    Disponible en: \url{https://stackoverflow.com/questions/40584424/simple-android-recyclerview-example}.

    \bibitem{stackoverflow}
    StackOverflow. \textit{Métodos PUT y GET con router.put y router.delete no funcionan (express)}. 
    Disponible en: \url{https://stackoverflow.com/questions/64842029/put-and-get-methods-with-router-put-and-router-delete-not-working-express}.

    \bibitem{stackoverflow}
    StackOverflow. \textit{Error "The requested module does not provide and export named default"}. 
    Disponible en: \url{https://stackoverflow.com/questions/71022803/the-requested-module-does-not-provide-an-export-named-default-error-but}.

    \bibitem{stackoverflow}
    StackOverflow. \textit{Como crear un contenedor rectangular alrededor de los elementos de una activity de Android Studio}. 
    Disponible en: \url{https://stackoverflow.com/questions/40582555/how-to-draw-a-rectangle-box-around-elements-of-android-activity}.

    \bibitem{stackoverflow}
    StackOverflow. \textit{Como crear un rectángulo con fondo transparente en Android Studio}. 
    Disponible en: \url{https://stackoverflow.com/questions/73084654/how-to-draw-rectangle-with-transparent-black-and-white-background-in-android-can}.

    \bibitem{stackoverflow}
    StackOverflow. \textit{Como cambiar la imagen de un ImageView al pulsar un botón en Android Studio}. 
    Disponible en: \url{https://stackoverflow.com/questions/24755849/android-imagebutton-how-to-change-the-image-when-button-is-clicked}.

    \bibitem{stackoverflow}
    StackOverflow. \textit{Como arreglar el error "Call requires API level 26 current min is 25" en Android Studio}. 
    Disponible en: \url{https://stackoverflow.com/questions/56695997/how-to-fix-call-requires-api-level-26-current-min-is-25-error-in-android}.

    \bibitem{stackoverflow}
    StackOverflow. \textit{Botón de Android Studio aparece en el editor gráfico pero no en el dispositivo en ejecución}. 
    Disponible en: \url{https://stackoverflow.com/questions/24116403/android-button-shows-in-graphical-layout-but-not-on-device}.
    
    \bibitem{stackoverflow}
    StackOverflow. \textit{Simple Android Alert Dialog}. 
    Disponible en: \url{https://stackoverflow.com/questions/26097513/android-simple-alert-dialog}.

    \bibitem{mongodb}
    MongoDB. \textit{Como empezar con el Driver de Java - Java Sync Driver - MondoDB Docs}. 
    Disponible en: \url{https://www.mongodb.com/docs/drivers/java/sync/current/get-started/#quick-start}.

    \bibitem{mongodb}
    MongoDB. \textit{Java y MongoDB}. 
    Disponible en: \url{https://www.mongodb.com/resources/languages/java}.

    \bibitem{mongodb}
    MongoDB. \textit{Uso de android studio y mongo DB realm - Cómo pushear un update a un Array en mi objeto}. 
    Disponible en: \url{https://www.mongodb.com/community/forums/t/using-android-studio-and-mongo-db-realm-how-to-push-an-update-to-an-array-in-my-object/201919}.

    \bibitem{sentry}
    Sentry Answers. \textit{Redondear un decimal a N posiciones en Java}. 
    Disponible en: \url{https://sentry.io/answers/round-a-number-to-n-decimal-places-in-java/}.

    \bibitem{futurestud}
    Future Stud. \textit{Retrofit2 como actualizar objetos en el servidor put vs patch}. 
    Disponible en: \url{https://futurestud.io/tutorials/retrofit-2-how-to-update-objects-on-the-server-put-vs-patch}.

    \bibitem{github}
    GitHub. \textit{Ejemplo de onClickListener}. 
    Disponible en: \url{https://gist.github.com/TheItachiUchiha/66322bc3a998bae23e56}.

    \bibitem{youtube}
    Youtube. \textit{SearchView con RecyclerView en Android Studio}. 
    Disponible en: \url{https://www.youtube.com/watch?v=tQ7V7iBg5zE}.

    \bibitem{chatgpt} OpenAI. ChatGPT \textit{Ayuda con retrofit en Android Studio} \url{https://chatgpt.com}
    \bibitem{chatgpt} OpenAI. ChatGPT \textit{Consulta sobre el reciclado de clases de Retrofit y clases de Java de Android Studio a JavaFX} \url{https://chatgpt.com}
    \bibitem{chatgpt} OpenAI. ChatGPT \textit{Ayuda con el paso de datos de una activity a otra en Android Studio} \url{https://chatgpt.com}
    \bibitem{chatgpt} OpenAI. ChatGPT \textit{Ayuda con la documentación en latex} \url{https://chatgpt.com}
    \bibitem{chatgpt} OpenAI. ChatGPT \textit{Como emplear Log.d para registrar logs en logcat} \url{https://chatgpt.com}

    \end{thebibliography}



\end{document} 
